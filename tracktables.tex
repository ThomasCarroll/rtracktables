%\VignetteIndexEntry{Visualising sequencing data using tracktables}
%\VignettePackage{tracktables}
%\VignetteEngine{knitr::knitr}

% To compile this document
% library('knitr'); rm(list=ls()); knit('tracktables.Rnw')

\documentclass[12pt]{article}\usepackage[]{graphicx}\usepackage[usenames,dvipsnames]{color}
%% maxwidth is the original width if it is less than linewidth
%% otherwise use linewidth (to make sure the graphics do not exceed the margin)
\makeatletter
\def\maxwidth{ %
  \ifdim\Gin@nat@width>\linewidth
    \linewidth
  \else
    \Gin@nat@width
  \fi
}
\makeatother

\definecolor{fgcolor}{rgb}{0.345, 0.345, 0.345}
\newcommand{\hlnum}[1]{\textcolor[rgb]{0.686,0.059,0.569}{#1}}%
\newcommand{\hlstr}[1]{\textcolor[rgb]{0.192,0.494,0.8}{#1}}%
\newcommand{\hlcom}[1]{\textcolor[rgb]{0.678,0.584,0.686}{\textit{#1}}}%
\newcommand{\hlopt}[1]{\textcolor[rgb]{0,0,0}{#1}}%
\newcommand{\hlstd}[1]{\textcolor[rgb]{0.345,0.345,0.345}{#1}}%
\newcommand{\hlkwa}[1]{\textcolor[rgb]{0.161,0.373,0.58}{\textbf{#1}}}%
\newcommand{\hlkwb}[1]{\textcolor[rgb]{0.69,0.353,0.396}{#1}}%
\newcommand{\hlkwc}[1]{\textcolor[rgb]{0.333,0.667,0.333}{#1}}%
\newcommand{\hlkwd}[1]{\textcolor[rgb]{0.737,0.353,0.396}{\textbf{#1}}}%

\usepackage{framed}
\makeatletter
\newenvironment{kframe}{%
 \def\at@end@of@kframe{}%
 \ifinner\ifhmode%
  \def\at@end@of@kframe{\end{minipage}}%
  \begin{minipage}{\columnwidth}%
 \fi\fi%
 \def\FrameCommand##1{\hskip\@totalleftmargin \hskip-\fboxsep
 \colorbox{shadecolor}{##1}\hskip-\fboxsep
     % There is no \\@totalrightmargin, so:
     \hskip-\linewidth \hskip-\@totalleftmargin \hskip\columnwidth}%
 \MakeFramed {\advance\hsize-\width
   \@totalleftmargin\z@ \linewidth\hsize
   \@setminipage}}%
 {\par\unskip\endMakeFramed%
 \at@end@of@kframe}
\makeatother

\definecolor{shadecolor}{rgb}{.97, .97, .97}
\definecolor{messagecolor}{rgb}{0, 0, 0}
\definecolor{warningcolor}{rgb}{1, 0, 1}
\definecolor{errorcolor}{rgb}{1, 0, 0}
\newenvironment{knitrout}{}{} % an empty environment to be redefined in TeX

\usepackage{alltt}




\RequirePackage{/Library/Frameworks/R.framework/Versions/3.1/Resources/library/BiocStyle/sty/Bioconductor}

\AtBeginDocument{\bibliographystyle{/Library/Frameworks/R.framework/Versions/3.1/Resources/library/BiocStyle/sty/unsrturl}}





\author{Thomas Carroll$^{1*}$\\[1em] \small{$^{1}$ Bioinformatics Facility, MRC Clincal Sciences Centre;} \\ \small{\texttt{$^*$thomas.carroll (at)imperial.ac.uk}}}

\title{Visualising ChIP-seq data -- tracktables}
\IfFileExists{upquote.sty}{\usepackage{upquote}}{}
\begin{document}

\maketitle

\begin{abstract}
 
 tracktables provides a set of tools to organise visualise your ChIP-seq data.
  \vspace{1em}
  
  \end{abstract}



\newpage

\tableofcontents

\section{Standard workflow}

\subsection{Quick start}

Tracktables has two main sets of functions to assist in rapid visualisation of genomics data. 

The first set of functions deal with creating IGV sessions XMLs, sample information files and HTML tables allowing for dynamic updating of IGV.

In this quick start we will look at building a simple HTML page of a ChIP-seq IGV session.

\begin{knitrout}
\definecolor{shadecolor}{rgb}{0.969, 0.969, 0.969}\color{fgcolor}\begin{kframe}
\begin{alltt}
\hlkwd{library}\hlstd{(tracktables)}
\hlstd{SampleSheet} \hlkwb{<-} \hlkwd{cbind}\hlstd{(}
  \hlkwd{c}\hlstd{(}\hlstr{"Pu1"}\hlstd{,}\hlstr{"Myc"}\hlstd{,}\hlstr{"Ik_prePro"}\hlstd{,}\hlstr{"Ik_pro"}\hlstd{),}
  \hlkwd{c}\hlstd{(}\hlstr{"ProB"}\hlstd{,}\hlstr{"ProB"}\hlstd{,}\hlstr{"PreProB"}\hlstd{,}\hlstr{"ProB"}\hlstd{),}
  \hlkwd{c}\hlstd{(}\hlstr{"Pu1"}\hlstd{,}\hlstr{"Myc"}\hlstd{,}\hlstr{"Ik"}\hlstd{,}\hlstr{"Ik"}\hlstd{)}
\hlstd{)}
\hlkwd{colnames}\hlstd{(SampleSheet)} \hlkwb{<-} \hlkwd{c}\hlstd{(}\hlstr{"SampleName"}\hlstd{,}\hlstr{'Tissue'}\hlstd{,}\hlstr{"Factor"}\hlstd{)}
\hlstd{fileSheet} \hlkwb{<-} \hlkwd{cbind}\hlstd{(}
  \hlkwd{c}\hlstd{(}\hlstr{"Pu1"}\hlstd{,}\hlstr{"Myc"}\hlstd{,}\hlstr{"Ik_prePro"}\hlstd{,}\hlstr{"Ik_pro"}\hlstd{),}
  \hlkwd{c}\hlstd{(}\hlnum{NA}\hlstd{,}\hlnum{NA}\hlstd{,}\hlnum{NA}\hlstd{,}\hlnum{NA}\hlstd{),}
  \hlkwd{c}\hlstd{(}\hlstr{"/Users/tcarroll/Downloads/randomTracks-2/Pu1DupMarkedNormalised.bw"}\hlstd{,}
    \hlstr{"/Users/tcarroll/Downloads/randomTracks-2/MycDupMarkedNormalised.bw"}\hlstd{,}
    \hlstr{"/Users/tcarroll/Downloads/randomTracks-2/Ikaros_2_preproBDupMarkedNormalised.bw"}\hlstd{,}
    \hlstr{"/Users/tcarroll/Downloads/randomTracks-2/Ikaros_1_proBDupMarkedNormalised.bw"}\hlstd{),}
  \hlkwd{c}\hlstd{(}\hlstr{"/Users/tcarroll/Downloads/randomTracks/Pu1_WithInput_Input_2_proB_peaks.bed"}\hlstd{,}
    \hlstr{"/Users/tcarroll/Downloads/randomTracks/Myc_WithInput_Input_Ch12_peaks.bed"}\hlstd{,}
    \hlstr{"/Users/tcarroll/Downloads/randomTracks/Ikaros_2_preproB_WithInput_Input_2_proB_peaks.bed"}\hlstd{,}
    \hlstr{"/Users/tcarroll/Downloads/randomTracks/Ikaros_1_proB_WithInput_Input_2_proB_peaks.bed"}\hlstd{)}
\hlstd{)}
\hlkwd{colnames}\hlstd{(fileSheet)} \hlkwb{<-} \hlkwd{c}\hlstd{(}\hlstr{"SampleName"}\hlstd{,}\hlstr{"bam"}\hlstd{,}\hlstr{"bigwig"}\hlstd{,}\hlstr{"interval"}\hlstd{)}
\end{alltt}
\end{kframe}
\end{knitrout}

\begin{knitrout}
\definecolor{shadecolor}{rgb}{0.969, 0.969, 0.969}\color{fgcolor}\begin{kframe}
\begin{alltt}
\hlkwd{library}\hlstd{(tracktables)}
\hlstd{SampleSheet} \hlkwb{<-} \hlkwd{cbind}\hlstd{(}
  \hlkwd{c}\hlstd{(}\hlstr{"Pu1"}\hlstd{,}\hlstr{"Myc"}\hlstd{,}\hlstr{"Ik_prePro"}\hlstd{,}\hlstr{"Ik_pro"}\hlstd{),}
  \hlkwd{c}\hlstd{(}\hlstr{"ProB"}\hlstd{,}\hlstr{"ProB"}\hlstd{,}\hlstr{"PreProB"}\hlstd{,}\hlstr{"ProB"}\hlstd{),}
  \hlkwd{c}\hlstd{(}\hlstr{"Pu1"}\hlstd{,}\hlstr{"Myc"}\hlstd{,}\hlstr{"Ik"}\hlstd{,}\hlstr{"Ik"}\hlstd{)}
\hlstd{)}
\hlkwd{colnames}\hlstd{(SampleSheet)} \hlkwb{<-} \hlkwd{c}\hlstd{(}\hlstr{"SampleName"}\hlstd{,}\hlstr{'Tissue'}\hlstd{,}\hlstr{"Factor"}\hlstd{)}
\hlstd{fileSheet} \hlkwb{<-} \hlkwd{cbind}\hlstd{(}
  \hlkwd{c}\hlstd{(}\hlstr{"Pu1"}\hlstd{,}\hlstr{"Myc"}\hlstd{,}\hlstr{"Ik_prePro"}\hlstd{,}\hlstr{"Ik_pro"}\hlstd{),}
  \hlkwd{c}\hlstd{(}\hlnum{NA}\hlstd{,}\hlnum{NA}\hlstd{,}\hlnum{NA}\hlstd{,}\hlnum{NA}\hlstd{),}
  \hlkwd{c}\hlstd{(}\hlstr{"/Users/tcarroll/Downloads/randomTracks-2/Pu1DupMarkedNormalised.bw"}\hlstd{,}
    \hlstr{"/Users/tcarroll/Downloads/randomTracks-2/MycDupMarkedNormalised.bw"}\hlstd{,}
    \hlstr{"/Users/tcarroll/Downloads/randomTracks-2/Ikaros_2_preproBDupMarkedNormalised.bw"}\hlstd{,}
    \hlstr{"/Users/tcarroll/Downloads/randomTracks-2/Ikaros_1_proBDupMarkedNormalised.bw"}\hlstd{),}
  \hlkwd{c}\hlstd{(}\hlstr{"/Users/tcarroll/Downloads/randomTracks/Pu1_WithInput_Input_2_proB_peaks.bed"}\hlstd{,}
    \hlstr{"/Users/tcarroll/Downloads/randomTracks/Myc_WithInput_Input_Ch12_peaks.bed"}\hlstd{,}
    \hlstr{"/Users/tcarroll/Downloads/randomTracks/Ikaros_2_preproB_WithInput_Input_2_proB_peaks.bed"}\hlstd{,}
    \hlstr{"/Users/tcarroll/Downloads/randomTracks/Ikaros_1_proB_WithInput_Input_2_proB_peaks.bed"}\hlstd{)}
\hlstd{)}
\hlkwd{colnames}\hlstd{(fileSheet)} \hlkwb{<-} \hlkwd{c}\hlstd{(}\hlstr{"SampleName"}\hlstd{,}\hlstr{"bam"}\hlstd{,}\hlstr{"bigwig"}\hlstd{,}\hlstr{"interval"}\hlstd{)}
\end{alltt}
\end{kframe}
\end{knitrout}

The first set of functions are for summarising and visualising coverage of genomic regions across samples and gene/genomic-interval sets.
The regionPlot function calculates normalised coverage across or over user specified regions to produce \Rclass{ChIPprofile} object.

\begin{knitrout}
\definecolor{shadecolor}{rgb}{0.969, 0.969, 0.969}\color{fgcolor}\begin{kframe}
\begin{alltt}
\hlkwd{print}\hlstd{(}\hlstr{"Coming Soon!"}\hlstd{)}
\end{alltt}
\end{kframe}
\end{knitrout}

\bibliography{library}

\end{document}
